\documentclass{article}
\usepackage[utf8]{inputenc}
\usepackage{enumitem}
\usepackage{amsmath,amssymb}



\title{ME 333 Homework 2}
\author{Marshall Johnson}
\date{January 18, 2022}

\begin{document}

\maketitle

\section*{Chapter 2 Exercises}

\begin{enumerate}[label=\textbf{\arabic*})]
    \setcounter{enumi}{2}
    \item \textbf{Describe the four functions that pin 12 of the PIC32MX795F512H can have. Is it 5 V tolerant?} \\
    
    Pin 12 of the PIC32MX795F512H has the following functions:
    \begin{enumerate}[label=\textbf{\alph*}.]
        \item AN4: Analog-to-digital input --- allows for monitoring of 
        analog voltage values (typically sensor inputs).
        \item C1IN-: Comparator negative input -- compares two analog input voltages
        and determines which is larger
        \item CN6: Change notification --- can generate an interrupt when input 
        voltage changes from high to low, or vice versa.
        \item RB4: Digital I/O --- allows for reading or output of a digital voltage
    \end{enumerate}

    \textbf{It is not 5V tolerant.}

    \item \textbf{Referring to the Data Sheet section on I/O Ports, what is the name of the SFR you have
    to modify if you want to change pins on PORTC from output to input?} \\

    SFR TRISC

    \item \textbf{The SFR CM1CON controls comparator behavior. Referring to the Memory
    Organization section of the Data Sheet, what is the reset value of CM1CON in
    hexadecimal?} \\

    0x00C3

    \pagebreak
    \item \textbf{In one sentence each, without going into detail, explain the basic function of the
    following items shown in the PIC32 architecture block diagram Figure 2.2: SYSCLK,
    PBCLK, PORTA to PORTG (and indicate which of these can be used for analog input on
    the NU32’s PIC32), Timer1 to Timer5, 10-bit ADC, PWM OC1-5, Data RAM, Program Flash Memory, 
    and Prefetch Cache Module.} \\

    \begin{enumerate}[label=\textbf{\alph*}.]
        \item SYSCLK: System clock; clocks CPU at maximum frequency of 80 MHz.
        \item PBCLK: Peripheral bus clock; Frequency set to SYSCLK divided by 1, 2, 4, or 8. Setting lower than SYSCLK's saves power.
        \item PORTA to PORTG: Registers that allow I/O pins to be accessed (bidirectional I/O ports) 
        (PORTA is missing on PIC32MX795F512H); Only PORTB can be used for analog input on NU32's PIC32.
        \item Timer1 to Timer5: Counts the number of pulses of a signal.
        \item 10-bit ADC: Analog-to-digital converter that can monitor up to 16 analog voltage values via 16 different pins.
        \item PWM OC1-5: Output compare pins generally used to generate PWM signals that can control motors or 
        create analog voltage output.
        \item Data RAM: Memory type that stores temporary data (128 KB).
        \item Program Flash Memory: More plentiful source of memory but slower to read and write (512 KB).
        \item Prefetch Cache Module: Stores recently executed program instructions, and can even run ahead 
        of current instruction to prefetch future instructions. 
    \end{enumerate}

    \item \textbf{List the peripherals that are not clocked by PBCLK.} \\
    
    \begin{itemize}
        \item USB
        \item CAN1, CAN2
        \item Ethernet
        \item DMAC
        \item ICD
    \end{itemize}

    \item \textbf{If the ADC is measuring values between 0 and 3.3 V, what is the largest voltage
    difference that it may not be able to detect? (It’s a 10-bit ADC.)} \\

    $3.3 V/2^{10} = 0.0032 V$

    \item \textbf{Refer to the Reference Manual chapter on the Prefetch Cache. What is the maximum
    size of a program loop, in bytes, that can be completely stored in the cache?} \\

    256 bytes

    \item \textbf{Explain why the path between flash memory and the prefetch cache module is 128 bits
    wide instead of 32, 64, or 256 bits.} \\

    The prefetch cache module stores recently executed instructions to ultimately improve performance
    by allowing the CPU quicker access than with flash memory. Instructions are placed in a 16-byte 
    wide prefetch cache buffer in preparation for execution. As such, each set of instructions occupies 
    128 bits of memory, which is why the path between flash memory and the prefetch cache module is 
    128 bits wide. This size of data path also provides the same bandwidth as a 32-bit path running 
    at 4x frequency.

    \item \textbf{Explain how a digital output could be confgured to swing between 0 and 4 V, even
    though the PIC32 is powered by 3.3 V.}

    An output pin can be configured as \textbf{open drain}. This increases the range of voltages
    the pin can produce by allowing the pin's transistor to pull voltage down to 0V (sink current) 
    or up to as high as 5.5V (turn off).  

    \item \textbf{PIC32’s have increased their flash and RAM over the years. What is the maximum
    amount of flash memory a PIC32 can have before the current choice of base addresses in
    the physical memory map (for RAM, flash, peripherals, and boot flash) would have to be
    changed? What is the maximum amount of RAM? Give your answers in bytes in
    hexadecimal.}

    \begin{itemize}
        \item Flash Memory: 41.9 MB | 0x02800000
        \item RAM: 486.5 MB | 0x1D000000
    \end{itemize}

    \item \textbf{Examine the Special Features section of the Data Sheet.} \\
    
    \begin{enumerate}[label=\textbf{\alph*}.]
        \item \textbf{If you want your PBCLK frequency to be half the frequency of SYSCLK, which bits
        of which Device Confguration Register do you have to modify? What values do you
        give those bits?} \\
        
        \begin{itemize}
            \item Bits 13-12 (FPBDIV$<$1:0$>$: Peripheral Bus Clock Divisor Default Value bits) of Register 28-2 (DEVCFG1)
            \item Set value to 01
        \end{itemize} 

        \item \textbf{Which bit(s) of which SFR set the watchdog timer to be enabled? Which bit(s) set
        the postscale that determines the time interval during which the watchdog must be
        reset to prevent it from restarting the PIC32? What values would you give these bits
        to enable the watchdog and to set the time interval to be the maximum?} \\

        \begin{itemize}
            \item Watchdog timer: Bit 23 of the FWDTEN SFR (Register 28-2) sets the watchdog timer to 
            be enabled
            \item Postscale: Bits 20-16 set the postscale that determines the time interval during 
            which the watchdog must be reset to prevent it from restarting the PIC32
            \item Set Bit 23 to 1 to enable the watchdog and Bits 20-16 to 10100 or greater to set
            the time interval to be the maximum (since combinations not shown in the data sheet 
            default to 10100).
        \end{itemize}

        \item \textbf{The SYSCLK for a PIC32 can be generated several ways, as discussed in the
        Oscillator chapter in the Reference Manual and the Oscillator Confguration section
        in the Data Sheet. The PIC32 on the NU32 uses the (external) primary oscillator in
        HS mode with the phase-locked loop (PLL) module. Which bits of which device
        confguration register enable the primary oscillator and turn on the PLL module?} \\

        \begin{itemize}
            \item Setting Bits 9-8 of Register 28-2 (DEVCFG1) to 10 will enable HS mode. 
            Bits 2-0 of the same register set to 011 will enable the primary oscilaltor with the 
            PLL module.
        \end{itemize}
        
    \end{enumerate}

    \item \textbf{Your NU32 board provides four power rails: GND, regulated 3.3 V, regulated 5 V, and
    the unregulated input voltage (e.g., 6 V). You plan to put a load from the 5 V output to
    ground. If the load is modeled as a resistor, what is the smallest resistance that would be
    safe? An approximate answer is fine. In a sentence, explain how you arrived at the
    answer} \\

    16-17 $\Omega$ would be the smallest resistance I would deem safe. Anything lower could cause overheating, 
    which would be best avoided. This range would allow for no more than about 300 mA draw from the NU32 --- 
    staying well below the 800 mA limit.

    \pagebreak
    \item \textbf{The NU32 could be powered by different voltages. Give a reasonable range of voltages
    that could be used, minimum to maximum, and explain the reason for the limits.}

    A reasonable voltage range to power the NU32 is 2.3 to 9 V. The NU32 requires at least 2.3 V, but also
    has a 5 V regulator to handle voltages above the ideal 2.3 to 3.6 V range. However, using a higher
    voltage than 9 V is not recommended, since the regulators will heat up. 

    \item \textbf{Two buttons and two LEDs are interfaced to the PIC32 on the NU32. Which pins are
    they connected to? Give the actual pin numbers, 1-64, as well as the name of the pin
    function as it is used on the NU32. For example, pin 37 on the PIC32MX795F512H
    could have the function D+ (USB data line) or RG2 (Port G digital input/output), but
    only one of these functions could be active at a given time.} \\

    \begin{itemize}
        \item LEDs: Pin 58 \& 59 --- RF0 (Digital I/O) and RF1 (Digital I/O), respectively
        \item Buttons: Pin 7 \& 55 --- $\overline{MCLR}$ (Master clear reset pin) and 
        RD7 (Digital I/O), respectively
    \end{itemize}


\end{enumerate}



\end{document}
