\documentclass{article}
\usepackage[utf8]{inputenc}
\usepackage{enumitem}
\usepackage{listings}
\usepackage[T1]{fontenc}
\usepackage{geometry}
\usepackage{amsmath}

\geometry{
 a4paper,
 left=20mm,
 right=20mm,
 top=20mm,
 bottom=20mm
 }

\lstset{
    language=C,
    showstringspaces=false,
    breaklines=true
}


\title{ME 333 Quiz 8 \\ [1ex] \large Brushed DC Motors}
\author{Marshall Johnson}
\date{March 3, 2022}

\begin{document}

\maketitle

\section*{Quiz 8}

\begin{enumerate}[label=\textbf{\arabic*})]
    \item \textbf{Write the motor power equation, derived in terms of V, $\tau$, I, $\omega$, L, and R: } \\

    $$\text{IV}_\text{motor} = \tau\omega + I^2R + LI\frac{dI}{dt}$$

    \item \textbf{Describe what each part of the equation in \#1 relates to:} \\
    
    \begin{flalign*}
        \tau\omega \rightarrow \text{mechanical power}&&        
    \end{flalign*}
    \begin{flalign*}
        I^2R \rightarrow \text{heat}&&        
    \end{flalign*}
    \begin{flalign*}
        LI\frac{dI}{dt} \rightarrow \text{power into (charging) or out of (discharging) inductor}&&        
    \end{flalign*}

    \item \textbf{Draw the speed-torque curve for a motor at 10V, with R = 10ohm, P = 10W, kt = 0.1Nm/A = 
    0.1Vs/rad, and label the max continuous region. } \\
    \\
    \\
    \\
    \\
    \\
    \\
    \\
    \\
    \\
    
    \item \textbf{Draw a motor connected to four switches, built in an h-bridge configuration. Label the switches 
    that would need to close to make the motor rotate in one direction (either CW or CCW) and the 
    switches to close to make the motor rotate in the other direction. } \\
 
    
\end{enumerate}

\end{document}
