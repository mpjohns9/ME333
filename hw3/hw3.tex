\documentclass{article}
\usepackage[utf8]{inputenc}
\usepackage{enumitem}
\usepackage{listings}
\usepackage{amsmath,amssymb}

\lstset{
    language=C,
    showstringspaces=false,
    breaklines=true
}



\title{ME 333 Homework 3}
\author{Marshall Johnson}
\date{January 25, 2022}

\begin{document}

\maketitle

\section*{Chapter 3 Exercises}

\begin{enumerate}[label=\textbf{\arabic*})]
    \item \textbf{Convert the following virtual addresses to physical addresses, and indicate whether the
    address is cacheable or not, and whether it resides in RAM, flash, SFRs, or boot flash.} \\

    \begin{enumerate}[label=\textbf{\alph*}.]
        \item \textbf{0x80000020} --- 0x00000020 | Cacheable | RAM
        \item \textbf{0xA0000020} --- 0x00000020 | Not Cacheable | RAM
        \item \textbf{0xBF800001} --- 0x0F800001 | Not Cacheable | SFRs
        \item \textbf{0x9FC00111} --- 0x1FC00111 | Cacheable | Boot Flash
        \item \textbf{0x9D001000} --- 0x1D001000 | Cacheable | Program Flash
    \end{enumerate}

    \setcounter{enumi}{2}
    \item \textbf{Refer to the Memory Organization section of the Data Sheet and Figure 2.1.} \\
    
    \begin{enumerate}[label=\textbf{\alph*}.]
        \item \textbf{Referring to the Data Sheet, indicate which bits, 0-31, can be used as input/outputs for
        each of Ports B through G. For the PIC32MX795F512H in Figure 2.1, indicate which
        pin corresponds to bit 0 of port E (this is referred to as RE0).} 

        \begin{itemize}
            \item Port B: 0-15
            \item Port C: 12-15
            \item Port D: 0-11
            \item Port E: 0-7
            \item Port F: 0-1; 3-5
            \item Port G: 2-3; 6-9
            \item Bit 0 of Port E = Pin 60
        \end{itemize}

        \pagebreak
        \item \textbf{The SFR INTCON refers to “interrupt control.” Which bits, 0-31, of this SFR are
        unimplemented? Of the bits that are implemented, give the numbers of the bits and
        their names.} \\
        \begin{itemize}
            \item Unimplemented: 5-7; 11; 13-15; 17-31
            \item Implemented:
            \begin{itemize}
                \item 0-4: INT0EP/INT1EP/INT2EP/INT3EP/INT4EP
                \item 8-10: TPC$<$2:0$>$
                \item 12: MVEC
                \item 16: SS0
            \end{itemize}
        \end{itemize}
    
    \end{enumerate}

    \setcounter{enumi}{6}
    \item \textbf{The processor.o file linked with your simplePIC project is much larger than your final
    .hex file. Explain how that is possible.} \\

    The processor.o file contains vitrual memory addresses for the PIC32's SFRs. This is a lot of information
    since multiple VAs can map to the same PA. After the linker takes object files and outputs a large .elf file, 
    the final step is to create a stripped down .hex file, which is significantly smaller.

    \item \textbf{The building of a typical PIC32 program makes use of a number of files in the XC32
    compiler distribution. Let us look at a few of them.} \\

    \begin{enumerate}[label=\textbf{\alph*}.]
        \item \textbf{Look at the assembly startup code pic32-libs/libpic32/startup/crt0.S. Although
        we are not studying assembly code, the comments help you understand what the
        startup code does. Based on the comments, you can see that this code clears the RAM
        addresses where uninitialized global variables are stored, for example. Find and list
        the line(s) of code that call the user’s main function when the C runtime startup
        completes.}

        Lines 522-526

        \item \textbf{Using the command xc32-nm -n processor.o, give the names and addresses of the
        five SFRs with the highest addresses.}
        \begin{itemize}
            \item bf88cb4c A C2FIFOCI31INV
            \item bfc02ff0 A DEVCFG3
            \item bfc02ff4 A DEVCFG2
            \item bfc02ff8 A DEVCFG1
            \item bfc02ffc A DEVCFG0
        \end{itemize}

        \pagebreak
        \item \textbf{Open the file p32mx795f512h.h and go to the declaration of the SFR SPI2STAT and its
        associated bit feld data type \_\_SPI2STATbits\_t. How many bit fields are defined?
        What are their names and sizes? Do these coincide with the Data Sheet?} \\

        10 bit fields defined (not including the empty fields): \\

        \textit{Name:Size}
        \begin{itemize}
            \item SPIRBF:1
            \item SPITBF:1
            \item SPITBE:1
            \item SPIRBE:1
            \item SPIROV:1
            \item SRMT:1
            \item SPITUR:1
            \item SPIBUSY:1
            \item TXBUFELM:5
            \item RXBUFELM:5
        \end{itemize}

        These coincide with the Data Sheet.
    \end{enumerate}

    \item \textbf{Give three C commands, using TRISDSET, TRISDCLR, and TRISDINV, that set bits 2
    and 3 of TRISD to 1, clear bits 1 and 5, and flip bits 0 and 4.}

    \begin{lstlisting}

        TRISDSET = 0b1100;

        TRISDCLR = 0b100010;

        TRISDINV = 0b10001;

    \end{lstlisting}

\end{enumerate}



\end{document}
