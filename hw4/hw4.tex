\documentclass{article}
\usepackage[utf8]{inputenc}
\usepackage{enumitem}
\usepackage{listings}
\usepackage{amsmath,amssymb}
\usepackage{geometry}

\geometry{
 a4paper,
 left=38mm,
 right=38mm,
 top=38mm,
 bottom=38mm
 }

\lstset{
    language=C,
    showstringspaces=false,
    breaklines=true
}



\title{ME 333 Homework 4}
\author{Marshall Johnson}
\date{January 31, 2022}

\begin{document}

\maketitle

\section*{Chapter 3 Exercises}

\begin{enumerate}[label=\textbf{\arabic*})]
    \setcounter{enumi}{3}
    \item \textbf{Modify simplePIC.c so that both lights are on or off at the same time, instead of opposite
    each other. Turn in only the code that changed.} \\

    See simplePIC\_simul.c

    \item \textbf{ Modify simplePIC.c so that the function delay takes an int cycles as an argument. The
    for loop in delay executes cycles times, not a fixed value of 1,000,000. Then modify main
    so that the first time it calls delay, it passes a value equal to MAXCYCLES. The next time it
    calls delay with a value decreased by DELTACYCLES, and so on, until the value is less than
    zero, at which time it resets the value to MAXCYCLES. Use \#define to define the constants
    MAXCYCLES as 1,000,000 and DELTACYCLES as 100,000. Turn in your code.} \\

    See simplePIC\_cycles.c

\end{enumerate}

\pagebreak
\section*{Chapter 4 Exercises}

\begin{enumerate}[label=\textbf{\arabic*})]
    \item \textbf{Identify which functions, constants, and global variables in NU32.c are private to NU32.c
    and which are meant to be used in other C fles.} \\

    Private to NU32.c:
    \begin{itemize}
        \item DEBUG
        \item FWDTEN
        \item WDTPS
        \item POSCMOD
        \item FNOSC
        \item FPLLMUL
        \item FPLLIDIV
        \item FPLLODIV
        \item FPBDIV
        \item UPLLEN
        \item UPLLIDIV
        \item FUSBIDIO
        \item FVBUSONIO
        \item FSOSCEN
        \item BWP
        \item ICESEL
        \item FCANIO
        \item FMIIEN
        \item FSRSSEL
        \item NU32\_DESIRED\_BAUD
    \end{itemize} 

    Meant for use in other files:
    \begin{itemize}
        \item NU32\_LED1\ LATFbits.LATF0
        \item NU32\_LED2\ LATFbits.LATF1
        \item NU32\_USER\ PORTDbits.RD7
        \item NU32\_SYS\_FREQ
        \item NU32\_Startup(void)
        \item NU32\_ReadUART3(char * string, int maxLength)
        \item NU32\_WriteUART3(const char * string)
    \end{itemize}

    \item \textbf{You will create your own libraries.} \\
    
    \begin{enumerate}[label=\textbf{\alph*}.]
        \item See invest.c
        \item Code: helper.c, helper.h, main\_2b.c \\
        
        All of the constant definitions, struct, and function prototypes were placed in the 
        helper.h file to separate them from the implementation. The functions are implemented
        in the helper.c file. This allows them to be accessed by users of the library,
        but the implementation of the functions is kept private. The MAX\_YEARS constant is 
        public, however, and can be modified by potential users. 
        \item Code: calculate.c, calculate.h, io.c, io.h, main\_2c.c \\
        
        Both io.c and calculate.c have their corresponding header files io.h and calculate.h.
        The functions that deal with input/output were placed in the io header file, with the 
        implementation in the c file. Similarly, the function dealing with calculations is 
        found in the calculations header file, with its implementation in the c file. The 
        calculations.h file includes the io.h file, so the MAX\_YEARS constant and Investment
        struct do not need to be redefined in the calculations header file. 
        

    \end{enumerate}

    \setcounter{enumi}{3}
    \item \textbf{Write a function, void LCD\_ClearLine(int ln), that clears a single line of the LCD (either
    line zero or line one). You can clear a line by writing enough space (’ ’) characters to
    fill it.} \\

    Modified LCDwrite.c file has been uploaded to Canvas with implementation of LCD\_ClearLine function. 
    Function is also included below:

    \begin{lstlisting}
        // Clears a line of the LCD screen
        void LCD_ClearLine(int ln) {
            char str[16] = "                ";

            if (ln == 1 || ln == 0) {

                LCD_Move(ln,0);
                LCD_WriteString(str); // clear desired column
                NU32_WriteUART3("\r\n");

            }
            else {
                NU32_WriteUART3("Invalid input line for clear.");
                return;
            }
    \end{lstlisting}


\end{enumerate}



\end{document}
