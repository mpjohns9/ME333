\documentclass{article}
\usepackage[utf8]{inputenc}
\usepackage{enumitem}
\usepackage{listings}
\usepackage{amsmath,amssymb}
\usepackage{geometry}
\usepackage[T1]{fontenc}


\geometry{
 a4paper,
 left=38mm,
 right=38mm,
 top=38mm,
 bottom=38mm
 }

\lstset{
    language=C,
    showstringspaces=false,
    breaklines=true
}



\title{ME 333 Homework 5}
\author{Marshall Johnson}
\date{February 11, 2022}

\begin{document}

\maketitle

\section*{Chapter 5 Exercises}

\begin{enumerate}[label=\textbf{\arabic*})]
    \setcounter{enumi}{2}
    \item \textbf{Answer the following questions.} \\
    
    \begin{enumerate}[label=\textbf{\alph*}.]
        \item \textbf{Which combinations of data types and arithmetic functions result in a jump to a
        subroutine? From your disassembly file, copy the C statement and the assembly
        commands it expands to (including the jump) for one example.} \\

        Data types and arithmetic functions that result in a jump to a subroutine: \\
        \begin{itemize}
            \item long long int
            \begin{itemize}
                \item[--] division
            \end{itemize}

            \item float
            \begin{itemize}
                \item[--] addition
                \item[--] subtraction
                \item[--] multiplication
                \item[--] division
            \end{itemize}

            \item long double
            \begin{itemize}
                \item[--] addition
                \item[--] subtraction
                \item[--] multiplication
                \item[--] division
            \end{itemize}
        \end{itemize}
        \vspace{1em}
        Example (long double/addition): \\
        \begin{lstlisting}
                d3 = d1+d2;
            9d003098: 8fc60040  lw a2,64(s8)
            9d00309c: 8fc70044  lw a3,68(s8)
            9d0030a0: 8fc40038  lw a0,56(s8)
            9d0030a4: 8fc5003c  lw a1,60(s8)
            9d0030a8: 0f4009c0  jal 9d002700 <__adddf3>
            9d0030ac: 00000000  nop
            9d0030b0: afc20060  sw v0,96(s8)
            9d0030b4: afc30064  sw v1,100(s8)
        \end{lstlisting}

        \item \textbf{For those statements that do not result in a jump to a subroutine, which
        combination(s) of data types and arithmetic functions result in the fewest assembly
        commands? From your disassembly, copy the C statement and its assembly
        commands for one of these examples. Is the smallest data type, char, involved in it?
        If not, what is the purpose of extra assembly command(s) for the char data type vs.
        the int data type?} \\

        \item int
        \begin{itemize}
            \item[--] addition
            \item[--] subtraction
            \item[--] multiplication
        \end{itemize}

        Example(int/addition): \\
        
        \begin{lstlisting}
            i3 = i1+i2;
        9d002f3c: 8fc30014  lw v1,20(s8)
        9d002f40: 8fc20018  lw v0,24(s8)
        9d002f44: 00621021  addu v0,v1,v0
        9d002f48: afc2004c  sw v0,76(s8)
        \end{lstlisting}

        The extra assembly command (andi) for the char data type is used to turn the 8 bit char into 
        the correct number of bits based on the CPU by performing an AND operation to add empty bits
        (in addition to the 8 bits that are desired).

        \item \textbf{Fill in the following table. Each cell should have two numbers: the number of
        assembly commands for the specified operation and data type, and the ratio of this
        number (greater than or equal to 1.0) to the smallest number of assembly commands
        in the table.} \\

        \begin{center}
            \begin{tabular}{ c | c | c | c | c | c }
              & char & int & long long & float & long double \\ 
             \hline
            + & 1.25 (5) & 1.0 (4) & 2.75 (11) & 1.25 (5) J & 2.0 (8) J \\  
            - & 1.25 (5) & 1.0 (4) & 2.75 (11) & 1.25 (5) J & 2.0 (8) J \\  
            * & 1.25 (5) & 1.0 (4) & 4.5 (18) & 1.25 (5) J & 2.0 (8) J \\ 
            / & 1.75 (7) & 1.75 (7) & 2.0 (8) J & 1.25 (5) J & 2.0 (8) J \\            
            \end{tabular}
        \end{center}
        \pagebreak
        \item \textbf{From the disassembly, find out the name of any math subroutine that has been added
        to your assembly code. Now create a map file of the program. Where are the math
        subroutines installed in virtual memory? Approximately how much program
        memory is used by each of the subroutines? You can use evidence from the
        disassembly file and/or the map file.} \\

        \begin{verbatim}
        Microchip PIC32 Memory-Usage Report

        kseg0 Program-Memory Usage
        section             address  length [bytes]      (dec)
        -------          ----------  -------------------------
        .text.dp32mul    0x9d001e00          0x4b8        1208   
        .text            0x9d0022b8          0x440        1088
        .text.dp32subadd 0x9d0026f8          0x430        1072   
        .text.dp32mul    0x9d002b28          0x32c         812   
        .text            0x9d002e54          0x5a8        1448
        .text.fpsubadd   0x9d0033fc          0x278         632   
        .text.fp32div    0x9d003674          0x230         560   
        .text.fp32mul    0x9d0038a4          0x1bc         444   
        \end{verbatim}
    \end{enumerate}

    \item \textbf{How many commands does each use?} \\
    
    \begin{enumerate}[label=\textbf{\alph*}.]
        \item u3 = u1 \& u2: 4 commands
        \item u3 = u1 $|$ u2: 4 commands
        \item u3 = u1 $<<$ u2: 3 commands
        \item u3 = u1 $>>$ u2: 3 commands
    \end{enumerate}


    

\end{enumerate}

\section*{Chapter 6 Exercises}

\begin{enumerate}[label=\textbf{\arabic*})]
    \item \textbf{Interrupts can be used to implement a fixed frequency control loop (e.g., 1 kHz). Another
    method for executing code at a fixed frequency is polling: you can keep checking the
    core timer, and when some number of ticks has passed, execute the control routine.
    Polling can also be used to check for changes on input pins and other events. Give pros
    and cons (if any) of using interrupts vs. polling.} \\ 

    Interrupts allow for instant notifcation when data is ready using a hardware mechanism. This prevents
    unecessary checking that occurs with polling. On the other hand, with polling, data timing is 
    predictable, whereas an interrupt can occur at any time.

    \pagebreak
    \setcounter{enumi}{3}
    \item 
    \begin{enumerate}[label=\textbf{\alph*}.]
        \item \textbf{What happens if an IRQ is generated for an ISR at priority level 4, subpriority level 2
        while the CPU is in normal execution (not executing an ISR)?} \\

        The ISR is executed immediately.
        

        \item \textbf{What happens if that IRQ is generated while the CPU is executing a priority level 2, 
        subpriority level 3 ISR?} \\

        The CPU jumps to the higher priority ISR (from the previous question) and executes before returning 
        to finish prcoessing the lower priority ISR.
        

        \item \textbf{What happens if that IRQ is generated while the CPU is executing a priority level 4,
        subpriority level 0 ISR?} \\

        Since they are the same priority level, the current ISR is finished before jumping to the priority 4,
        subpriority 2 interrupt from the first question.


        \item \textbf{What happens if that IRQ is generated while the CPU is
        executing a priority level 6, subpriority level 0 ISR?} \\

        The CPU finishes processing the priority level 6 ISR before jumping to and executing the ISR from
        the first question (priority 4.2).


    \end{enumerate}

    \item \textbf{An interrupt asks the CPU to stop what it’s doing, attend to something else, and then
    return to what it was doing. When the CPU is asked to stop what it’s doing, it needs to
    remember “context” of what it was working on, i.e., the values currently stored in the
    CPU registers.} \\

    \begin{enumerate}[label=\textbf{\alph*}.]
        \item \textbf{Assuming no shadow register set, what is the first thing the CPU must
        do before executing the ISR and the last thing it must do upon completing the ISR?} \\

        Before executing the ISR, the CPU must save the contents of the CPU registers to RAM. 
        After completing the ISR, must copy that data (the context) back from RAM to its 
        registers. This restores the CPU state prior to executing the ISR, allowing it to
        continue from where it left off. 

        \item \textbf{How does using the shadow register set change the situation?} \\
        
        The shadow register set (SRS) is an extra full set of registers. Rather than saving 
        and restoring, the CPU can switch to this extra register set when processing the ISR. 
        After finishing the ISR, it can switch back to the original register set without the 
        need to restore, as it will be in the same state it was before switching to the SRS.
    \end{enumerate}

    \pagebreak
    \setcounter{enumi}{7}
    \item \textbf{For the problems below, use only the SFRs IECx, IFSx, IPCy, and INTCON, and their
    CLR, SET, and INV registers (do not use other registers, nor the bit fields as in
    IFS0bits.INT0IF). Give valid C bit manipulation commands to perform the operations
    without changing any uninvolved bits. Also indicate, in English, what you are trying to
    do, in case you have the right idea but wrong C statements. Do not use any constants
    defined in Microchip XC32 files; just use numbers.} \\

    \begin{enumerate}[label=\textbf{\alph*}.]
        \item \textbf{Enable the Timer2 interrupt, set its flag status to 0, and set its vector’s priority and
        subpriority to 5 and 2, respectively.}

        \begin{lstlisting}
            // Set IEC0 bit 8 to 1 (enable interrupt)
            IEC0SET = 0x100; 

            // Clear IFS0 bit 8 to 0 (set flag status)
            IFS0CLR = 0x100; 

            // Set IPC2 bits 2 and 4 to 1 (vector priority 5)
            IPC2SET = 0x14; 

            // Set IPC2 bits 0-1 to 1 (subpriority 2)
            IPC2SET = 0x3; 
        \end{lstlisting}

        \item \textbf{Enable the Real-Time Clock and Calendar interrupt, set its flag status to 0, and set its
        vector’s priority and subpriority to 6 and 1, respectively.} \\

        \begin{lstlisting}
            // Set IEC1 bit 15 to 1 (enable interrupt)
            IEC1SET = 0x8000; 

            // Clear IFS1 bit 15 to 0 (set flag status)
            IFS1CLR = 0x8000; 

            // Set IPC8 bits 27 and 28 to 1 (vector priority 6)
            IPC8SET = 0x18; 

            // Set IPC8 bit 24 to 1 (subpriority 1)
            IPC8SET = 0x1000000; 
        \end{lstlisting}

        \pagebreak
        \item \textbf{Enable the UART4 receiver interrupt, set its flag status to 0, and set its vector’s
        priority and subpriority to 7 and 3, respectively.} \\

        \begin{lstlisting}
            // Set IEC2 bit 4 to 1 (enable interrupt)
            IEC2SET = 0x10; 

            // Clear IFS2 bit 4 to 0 (set flag status)
            IFS2CLR = 0x10; 

            // Set IPC12 bits 10-12 to 1 (vector priority 7)
            IPC12SET = 0x1c00; 

            // Set IPC8 bits 8 and 9 to 1 (subpriority 3)
            IPC12SET = 0x300; 
        \end{lstlisting}

        \item \textbf{Enable the INT2 external input interrupt, set its flag status to 0, set its vector’s
        priority and subpriority to 3 and 2, and configure it to trigger on a rising edge.} \\

        \begin{lstlisting}
            // Set IEC0 bit 11 to 1 (enable interrupt)
            IEC0SET = 0x800; 

            // Clear IFS0 bit 11 to 0 (set flag status)
            IFS0CLR = 0x800; 

            // Set IPC2 bits 26 and 27 to 1 (vector priority 3)
            IPC12SET = 0xc000000; 

            // Set IPC8 bit 25 to 1 (subpriority 2)
            IPC12SET = 0x2000000; 
        \end{lstlisting}
        
    \end{enumerate}

    \pagebreak
    \item \textbf{Edit Code Sample 6.3 so that each line correctly uses the “bits” forms of the SFRs.
    In other words, the left-hand sides of the statements should use a form similar to that
    used in step 5, except using INTCONbits, IPC0bits, and IEC0bits.} \\

    The following lines of code were extracted from Code Sample 6.3 and modified to use the "bits" forms
    of the SFRs. For the complete version of modified code, please see sample\_6-3.c.

    \begin{lstlisting}
        // clear the interrupt flag
        IFS0bits.INT0IF = 0;   
        // clear the interrupt flag 
        IFS0bits.T2IF = 0;      
        // step 3: INT0 and INT1 trigger on rising edge
        INTCONbits.INT0EP = 1; 
        INTCONbits.INT1EP = 1;                  
        // step 4: clear 5 priority and subp bits for INT0
        IPC0bits.INT0IP = 0;    
        IPC0bits.INT0IS = 0;   
        // step 4: set INT0 to priority 6 subpriority 0
        IPC0bits.INT0IP = 6;    
        IPC0bits.INT0IS = 0;    
        // step 4: clear 5 priority and subp bits for INT1
        IPC1bits.INT1IP = 0;    
        IPC1bits.INT1IS = 0;    
        // step 4: set INT1 to priority 6 subpriority 0
        IPC1bits.INT1IP = 6;    
        IPC1bits.INT1IS = 0;   
        // step 6: enable INT0 and INT1 interrupts 
        IEC0bits.INT0IE = 1;    
        IEC0bits.INT1IE = 1;
    \end{lstlisting}

    \setcounter{enumi}{15}
    \item \textbf{Modify Code Sample 6.2 so the USER button is debounced. How can you change the
    ISR so the LEDs do not flash if the falling edge comes at the beginning of a very brief,
    spurious down pulse? Verify that your solution works.} \\

    Delay for 10 ms (duration of real button press), then check value of USER button. If pressed (false), 
    execute code to turn LEDs on and off. Otherwise, just clear the interrupt flag.

    See debounce.c for code.

    \item \textbf{Using your solution for debouncing the USER button (Exercise 16), write a stopwatch
    program using an ISR based on INT2.} \\

    See timer.c.

\end{enumerate}



\end{document}
